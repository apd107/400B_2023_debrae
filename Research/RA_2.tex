
\documentclass[fleqn,usenatbib]{mnras}

\usepackage{newtxtext,newtxmath}
\usepackage[T1]{fontenc}

\DeclareRobustCommand{\VAN}[3]{#2}
\let\VANthebibliography\thebibliography
\def\thebibliography{\DeclareRobustCommand{\VAN}[3]{##3}\VANthebibliography}


%%%%% AUTHORS - PLACE YOUR OWN PACKAGES HERE %%%%%


\usepackage{graphicx}
\usepackage{amsmath}

\title{Tidal Evolution of M33’s Dark Matter Halo}

\author{
Aidan DeBrae$^{1}$
\\
% List of institutions
$^{1}$Department of Astronomy, University of Arizona, 933 North Cherry Avenue, Tucson, AZ, 85719}

% Don't change these lines
\begin{document}
\label{firstpage}
\pagerange{\pageref{firstpage}--\pageref{lastpage}}
\maketitle

% Select between one and six entries from the list of approved keywords.
% Don't make up new ones.
\begin{keywords}
Galaxy evolution -- local group -- dark matter halo -- tidal affects
\end{keywords}

%%%%%%%%%%%%%%%%%%%%%%%%%%%%%%%%%%%%%%%%%%%%%%%%%%

%%%%%%%%%%%%%%%%% BODY OF PAPER %%%%%%%%%%%%%%%%%%

\section{Introduction}

In roughly 4 billion years the Local Group will experience a catastrophic event. That event is when our own galaxy, the Milky Way (MW), will collide with our closest neighboring galaxy, Andromeda (M31). The collision of these two galaxies will produce immense changes to their own structure as well as any nearby satellite galaxies. The interest of this research paper focuses on the evolution of M31's most massive satellite galaxy, Triangulum (M33). More specifically, the research will investigate the tidal evolution of M33's dark matter halo. The primary physical features which are of great importance during it's evolution are the changes to it's internal dark matter profile and the mass loss of the dark matter halo. 

The purpose of this research is to develop a better understanding of the evolution of dark matter in a dynamically changing universe. Astronomers strive to understand how dark matter behaves as it plays a very important role in the structure of our universe and especially how galaxies evolve. We've come to understand that a universe which is entirely composed of baryonic matter does not fit within our current observational data. Evidence suggests an unknown and highly elusive form of matter influences the ways in which galaxies behave. The primary evidence which leads to this conclusion is that the rotation curve of galaxies does not match previous dynamical models \citep{Hoeneisen19}. Astrophysicists suggest there exists a ``hidden'' source of matter which contributes to the orbital dynamics of the galaxy, dark matter. Our comprehension and investigation of dark matter produces a sort of trickle down affect. Observations of the rotation curves of galaxies suggest the presence of dark matter. Current theories attempt to predict how dark matter halos might influence the physical features of a galaxy. In turn, scientists are deeply interested in developing models of dark matter halos in hopes of constructing a more clear image of their behavior and we might detect them. To accurately build these models it is vital to reconstruct the exact environment in which the halo of interest exists within. This is of great importance because the dark matter halos of satellite galaxies can affect the  For example, astrophysicists have ran simulations modelling the halo mass of MW \& M33 which suggest their satellite galaxies greatly influence the final calculation of their mass \citep{Patel17}. Thus, understanding the characteristics of M33's dark matter halo after the galactic collision could provide insight into how the dark matter structure of the local group affects galaxy evolution.

The theoretical research investigating the interaction and tidal affects of M33 on it's host halo, M31, have shed light on the importance of this system. As stated previously, astrophysicists have used simulations to calculate the theoretical mass of the dark matter halos of MW \& M33. Using a Bayesian inference scheme which combines the satellite halo's position, velocities, and specific orbital angular momenta, researchers concluded that the resulting host halo mass is more susceptible to bias when using measurements of the current position and velocity of the satellites, especially when satellites are at short-lived phases of their orbits \citep{Patel17}.
\begin{figure}
	\includegraphics[width=\columnwidth]{host-satellite.png}
    \label{fig:host-satellite}
\end{figure}
This conclusion was accentuated by the fact that researchers previously determined that simulations suggest M33 is completing it's first orbit around M31 and is heading toward it's pericentric approach \citep{Patel16}. Thus, evidence is accumulating that points toward M33 having significant influence over the physical features of M31 and it's dark matter halo in the coming future. Additionally, astrophysicists have created hydro-dynamical models of the interaction between M31 \& M33 and have successfully reproduced characteristics observed in the system such as the spiral arm structure, stellar streams, and even a star formation burst \citep{Semczuk18}. It becomes clear that ongoing research supports the fact that to better understand the nature of a galaxy's dark matter halo it is necessary to account for the affects due to any massive satellite halos. Not only does the evidence suggest that the presence of a massive satellite halo will influence things such as the host halo mass but it also has the potential to explain certain characteristics of the baryonic matter observed in the system. This motivates the interest of this research paper which aims to investigate the tidal evolution of M33's dark matter halo in the event of a galactic collision. Its possible that learning more about the changes in tidal forces between the dark matter halos during an extreme event could offer more insight into how the galactic system will interact in the near future. 

It is important to explore a nearby host-satellite dark matter halo system such as M33 \& M31 because we currently lack understanding of dark matter and how it might affect galaxy evolution. Scientists are currently unable to explain the density profile of these dark matter halos. It then becomes very difficult to accurately predict how the dark matter halo is structured around the galaxy. According to evidence presented earlier, the tidal evolution of M33 during the galactic collision could help constrain what we might expect for a halo density profile. It's possible it could even help predict what sort of observational data we might receive according to the tidal affects of M33's dark matter halo. Furthermore, analyzing the shape of the dark matter halo due to tidal affects could also help us understand how a dark matter halo might be structured. 

\section{Proposal}
\subsection{Questions to be Addressed}
First, I'd like to address the question "What is the shape of the dark matter distribution of M33? How does this change with time? Is it elongated/ellipsoid or spherical? What do terms like prolate, oblate, or triaxial halos mean?" I think another question which compliments this investigation is "What is the time evolution of the inner dark matter density profile of M33? Does it become more or less concentrated with time? Is it well fit by a Hernquist profile? What might it mean if there is or isn’t evolution?" If time allows, I'd like to also answer this question.
\subsection{Approach/Methodology}
I think the best approach to answering the first question involving the shape of the dark matter halo is reflected in lab 7. This lab demonstrates how a histogram and contour fitting can illustrate the shape of a astronomical object. For my project, instead of a 2d histogram model I will use a 3d model. This will be accomplished by extracting particle position data from the simulation. Of course, to analyze the evolution of the halo, I'll create many plots each according to the red-shift values as MW \& M31 approach each other and collide. Various pieces of code which we have developed over the course of the semester will prove useful such as the code which calculates the center of mass of a given halo. To answer the second I can apply some of the code that will be used to answer the first. I can calculate the density profile of the particles within the halo according to it's time evolution from the particle data extracted in the first steps of my project. Using code similar to the exercises conducted in homework 5 and lab 6 will prove to be useful for understanding the density profile. In homework 5 we define the Hernquist mass profile which will be very useful in calculating the Hernquist density profile. Lab 6 provides an example of analyzing the bulge density across a given axis. I can plot the density of particles across each axis of the halo and analyze its properties. 
\begin{figure}
	\includegraphics[width=\columnwidth]{density_code.png}
    \caption{Example of the piece of code which creates a 2D density contour diagram of M31}
    \label{fig:density_code}
\end{figure}
\begin{figure}
	\includegraphics[width=\columnwidth]{density_contour.png}
    \caption{Result of the density contour plot of M31}
    \label{fig:density_figure}
\end{figure}
\begin{figure}
	\includegraphics[width=\columnwidth]{hernquist.png}
    \caption{Example of how the Hernquist density profile might be calculated}
    \label{fig:Hernquist}
\end{figure}
I believe the above figures provide a strong foundation in how I will complete the investigation into the questions I've chosen. I want to mention that none of the above code will be followed exactly. They are presented as a means for guiding how I will create my own code. This is especially true of the Hernquist mass profile from homework 5. 
\subsection{Hypothesis}
I believe that M33's dark matter halo will start with a semi-spheroidal shape with a Hernquist density profile. As the halo evolves, I think it will begin to exhibit an oblate shape as the gravitational pull from tidal disruption caused by it's host halo and MW approaching one another. This will inevitably cause the density profile to change in which the density will become more concentrated toward the combined MW \& M31 dark matter halo. I believe the immense gravitational contribution from MW \& M31 will change the shape an density profile of M33. If it does not change that could mean that M33 is outside of the range of the tidal affects of their collision.   

\bibliographystyle{mnras}
\bibliography{RA_2_references} 



% Don't change these lines
\bsp	% typesetting comment
\label{lastpage}
\end{document}

% End of mnras_template.tex
